% This file is part of the Platypus project.
% Copyright 2016 the authors.

% to-do
% - finish writing MKN's introduction
% - write data and cannon section
% - write method and results section
% - write discussion section
% - put in references
% - cite codes!
% - put in acknowledgement
% - forward to Freeman, others?
% - submit to journal and arXiv

\documentclass[12pt, letterpaper, preprint]{aastex}

\newcommand{\acronym}[1]{{\small{#1}}}
\newcommand{\project}[1]{\textsl{#1}}
\newcommand{\apogee}{\acronym{\project{APOGEE}}}
\newcommand{\thecannon}{\project{The~Cannon}}
\newcommand{\foreign}[1]{\textsl{#1}}
\newcommand{\eg}{\foreign{e.\,g.}}
\newcommand{\etal}{\foreign{et\,al.}}

\begin{document}

\title{Chemical tagging: \\
       Identification of stellar phase-space structures \\
       purely by chemical-abundance similarity}
\author{David~W.~Hogg\altaffilmark{1,2,3,4},
        Andrew~R.~Casey\altaffilmark{5},
        M.~Ness\altaffilmark{4},
        Hans-Walter~Rix\altaffilmark{4}, \&
        Daniel~Foreman-Mackey\altaffilmark{6}}
\altaffiltext{1}{SCDA}
\altaffiltext{2}{Center for Cosmology and Particle Physics, Department of Phyics,
  New York University, 4 Washington Pl., room 424, New York, NY, 10003, USA}
\altaffiltext{3}{Center for Data Science, New York University, 726 Broadway, 7th Floor, New York, NY 10003, USA}
\altaffiltext{4}{Max-Planck-Institut f\"ur Astronomie, K\"onigstuhl 17, D-69117 Heidelberg, Germany}
\altaffiltext{5}{IoA}
\altaffiltext{6}{UW}
\email{david.hogg@nyu.edu}

\begin{abstract}
% Context:
The chemical-tagging hypothesis is that every star ought to carry---%
in its detailed chemical abundances---%
an informative signature of its molecular-cloud birth location and time.
This idea has not previously yielded scientific successes, probably because of the
noise in chemical-abundance measurements.
However, we have succeeded in improving vastly spectroscopic measurements with \thecannon\ 
(now delivering individual abundances with precisions around 0.04-dex).
% Aims:
We test the chemical-tagging hypothesis by looking at clusters in abundance space
and confirming that they are clustered in phase space.
% Methods:
We identify (by the k-means algorithm) overdensities of stars observed by the \apogee\ spectroscopic survey
in a 15-dimensional chemical-abundance space delivered by \thecannon,
and plot the associated stars in phase space.
We use \emph{only} abundance-space information (no positional information) to identify stellar groups.
% Results:
We find that clusters in abundance space are indeed clusters in phase space.
We recover known clusters and stellar streams.
This confirms the chemical-tagging hypothesis and verifies the precision of abundances delivered by \thecannon.
This is the first-ever project to identify phase-space structures by blind search in abundance space;
the prospects for future data sets are very good.
\end{abstract}

\keywords{
  Galaxy: abundances
  ---
  Galaxy: stellar content
  ---
  Galaxy: structure
  ---
  globular clusters: general
  ---
  open clusters and associations: general
  ---
  stars: abundances
}

\section{Introduction}

\section{Introduction}\label{sec:Intro}

Freeman and Bland-Hawthorn cite Weinberg (1997) in the opening of
their 2002 review on `The New Galaxy: Signatures of Formation', with
the assertion that while much progress has been made since 1977, the
theory galaxy formation remains a major outstanding problem.
However, we are now in a new domain in stellar astronomy; one where it
is possible to test the principles of galactic archeology and chemical
tagging, precisely as laid out by Freeman and Bland-Hawthorn.
Galactic archeology describes the approach of exploring the formation
history of the Galaxy via examining the detailed chemical abundances
of its stars.
Within this realm lies the prospect of chemical tagging, which
describes associating now spatially incoherent stellar groups which
have built up the Milky Way, via their identical chemical
compositions.
These prospects rely on the Milky Way's bulge, disk and halo being
comprised of numerous small satellite systems, which have built up our
galaxy over time and which have formed from the same molecular gas and
so are chemically indistinguishable \citep[see][and references
  therein]{DaSilva2015}.
The systems that have formed the Galaxy are expected to no longer be
spatially coherent, having been dispersed throughout the galaxy via
processes of mixing and radial migration \citep[\eg,][]{Roskar2009, Quillen2015}.
Therefore, the chemical phase space of the stars is the tool we have
to reconstruct the building blocks of our galaxy at this snapshot in
time.
By exploring the multi chemical abundance information of Milky Way
stars (for many elements, $>15$), the hope is that stellar siblings
can be identified and that the dimensionality of this space can serve
as fingerprints of the stars.

The prospects to test the principle that the idea that the Milky Way
has been built up via multiple small satellite systems that are
chemically distinct can now be put to the test with the multitude of
data available from high resolution spectroscopic surveys.
These surveys are systematically mapping the stellar content of the
Milky Way and its chemical and dynamical phase space.
The detailed characterisation of the Milky Way's chemical phase space
are the founding principles and goals of the major spectroscopic
surveys of the current era, including GALAH \citep{Freeman2013, da
  Silva 2015}, Gaia-ESO \citep{Gilmore 2012} and APOGEE
\citep{Majewsk2015}.
These surveys are measuring high resolution (R $=$ 20,000-50,000),
high signal to noise spectra (SNR $>$ 100) for hundreds of thousands
of stars across a very wide spatial extent of the Milky Way's disk,
bulge and halo.
With this multitude of large stellar surveys underway, stars are now
exceptionally powerful tools that can be used to reconstruct the
galaxy's formation and evolution history.

A major limitation is now not the data quantify or quality, but the
current precision on abundance measurements \citep[\eg,][]{Ting2015,
  Martel2015} and the traditional methodologies employed, which do not
exploit the full information content nor dimensionality of the
data.
Achieving higher precision and exploiting the entire information
content of the data is critical if we are to reconstruct the chemical
components of the Galaxy and for chemical tagging to succeed.
Attempts to date to identify the building blocks of the Milky Way via
chemical tagging have relied on current abundance precisions and
explored different approaches.
These have not successfully identified components of the galaxy's
assembly that had distinct formation sites (\eg, Ting et al., 2015, Mitschang. et al. 2013,
Blanco-Cuaresma2015).
However, they have highlighted the need for higher precision and new
approaches for exploiting the data $<$things that are good things that
are bad$>$.
Blanco-Cuaresma et al., 2015 demonstrate that there is a large degree
of overlap when exploring the chemical space of known open clusters
which challenges chemically tagging and separating them, but also note
that there is room for improvement if more elements were included in
their analysis and models improved.
Ting et al., 2015 presented an algorithm to detect groups of
chemically homogeneous stars which they reported as challenging and
with a low signal to noise in the APOGEE dataset.
Mitschang et al., 2013 propose that chemical tagging may be able to
identify coeval groups of stars but these are not necessarily from a
distinct origin of formation.

The by-hand study of solar twins, for a handful of stars, has shown
however that the chemical information in some elements offers powerful
signatures for chemical tagging (Melendez2014, Jofre2015) and PCA
approach of Ting et al., 2014 shows there are many independent
dimensions of information contained in the multi abundance phase space
being delivered by the large spectroscopic surveys (e.g. Ting et al.,
2014 report a dimensionality of 7 for GALAH survey).
These approaches are important to establish which elements should be
targeted in surveys.

DWH:  We successfully achieve chemical tagging and we do so because we are
working with a higher precision that previously and exploit the full
information in the data....

DWH: With \thecannon\ we achieve both higher precision and exploit the full
information content of the data.....cite Casey et al., 2016 Ness et
al., 2016 in prep = do you want me to write things about this...

DWH: We do this with the APOGEE survey which covers a much wider
spatial extent but GALAH will be much more powerful as they will
report 29 elemental abundances which carry a larger quantity of
nucleosynthetic information...

DWH: Open clusters look one-metal (CITE); globulars look peculiar
(CITE); but no-one has ever gone the other way!...

DWH: Fast intro to \apogee.

DWH: Fast intro to \thecannon.

DWH: BASH the \apogee\ discontiguous field positioning.

DWH: Think of this as a hypothesis test.

\section{Sample and abundances}

MKN: Details about \apogee\ data.

AC: Details about \thecannon\ and implementation.

DWH: Figure showing the full data set; discuss structure.

\section{Method and results}

Although only hints of small-scale clustering in the abundance space
are visible in \figurename~DWH, exploration of the data indicates that
known clusters do appear in the abundance space as over-densities.
This encourages us to look for over-densities automatically in the
abundance space and see if anything found that way would be over-dense
in phase space.
The simplest method for clustering points in $D$-dimensional space is
the \emph{k-means} algorithm (DWH CITE).
This algorithm is fast and performs well in practice in problems of
this nature.

DWH: Scaling of the data
DWH: number of clusters
DWH: restarts for local minima
DWH: Citation out to numpy and sklearn.

DWH: Reminder that we don't need the absolute best clustering to
demonstrate the hypothesis.

DWH: Visualize densest clusters.  Some of them are known clusters.
Show examples.

DWH: Do we have streams or other structures?

DWH: What have we demonstrated?

\section{Discussion}

DWH: For the first time, we see small-scale informative features in abundance space.

DWH: When we identify the small-scale features in abundance space, they are informative in position space.

DWH: How do we show that we aren't just getting clusters because \apogee\ has plate-to-plate calibration variations?

DWH: Chemical tagging will work.

\acknowledgements
DWH: It is a pleasure to thank...

DWH: grants and grant numbers.

DWH: SDSS-III and SDSS-IV.

DWH: NASA ADS

DWH: github repo and hash.

DWH: Be sure to cite software a la the new ApJ guidelines.

\end{document}
