% This file is part of the Platypus project.
% Copyright 2016 the authors.

\documentclass[12pt, letterpaper, preprint]{aastex}

\newcommand{\acronym}[1]{{\small{#1}}}
\newcommand{\project}[1]{\textsl{#1}}
\newcommand{\apogee}{\acronym{\project{APOGEE}}}
\newcommand{\thecannon}{\project{The~Cannon}}

\begin{document}

\title{Identifying stellar clusters in chemical-abundance space}
\author{DWH, MKN, AC, HWR, others}

\begin{abstract}
% Context:
There is a dusty, old, and un-realized idea (dubbed ``chemical tagging'')
that every star ought to carry---%
in its detailed chemical abundances---%
an informative signature of its molecular-cloud birth location and time.
This idea is pretty, but has never yielded scientific fruit.
While astronomers have become more jaded, we have quietly been making
spectroscopic measurements of chemical abundances far more precise with \thecannon\ 
(now delivering 0.03-dex-level precision).
% Aims:
The chemical-tagging hypothesis is tested by looking at clusters in abundance space
and confirming that they are clustered in phase space.
% Methods:
We identify (by eye) overdensities of stars observed by the \apogee\ spectroscopic survey
in a 15-dimensional chemical-abundance space delivered by \thecannon,
and plot their locations in phase-space.
We use \emph{only} abundance-space information (no positional information) to identify stellar groups.
We don't know full 6-d phase-space positions;
we only have precise angular positions and radial velocities, and imprecise distances.
% Results:
We find that clusters in abundance space are indeed clusters in position space.
This confirms the chemical-tagging hypothesis and
verifies the precision of abundances delivered by \thecannon.
This is the first-ever project to identify stellar clusters by blind search in abundance space.
\end{abstract}

Hello World!!

\end{document}
