% This file is part of the Platypus project.
% Copyright 2016 the authors.

% to-do
% - finish writing MKN's introduction
% - write data and cannon section
% - write method and results section
% - write discussion section
% - put in references
% - cite codes!
% - put in acknowledgement
% - forward to Freeman, others?
% - submit to journal and arXiv

\documentclass[12pt, letterpaper, preprint]{aastex}

\newcommand{\acronym}[1]{{\small{#1}}}
\newcommand{\project}[1]{\textsl{#1}}
\newcommand{\apogee}{\project{\acronym{APOGEE}}}
\newcommand{\gaiaeso}{\project{Gaia--\acronym{ESO}}}
\newcommand{\galah}{\project{\acronym{GALAH}}}
\newcommand{\thecannon}{\project{The~Cannon}}
\newcommand{\foreign}[1]{\textsl{#1}}
\newcommand{\eg}{\foreign{e.\,g.}}
\newcommand{\etal}{\foreign{et\,al.}}

\begin{document}

\title{Chemical tagging: \\
       Identification of stellar phase-space structures \\
       purely by chemical-abundance similarity}
\author{David~W.~Hogg\altaffilmark{1,2,3,4},
        Andrew~R.~Casey\altaffilmark{5},
        M.~Ness\altaffilmark{4},
        Hans-Walter~Rix\altaffilmark{4}, \&
        Daniel~Foreman-Mackey\altaffilmark{6}}
\altaffiltext{1}{SCDA}
\altaffiltext{2}{Center for Cosmology and Particle Physics, Department of Phyics,
  New York University, 4 Washington Pl., room 424, New York, NY, 10003, USA}
\altaffiltext{3}{Center for Data Science, New York University, 726 Broadway, 7th Floor, New York, NY 10003, USA}
\altaffiltext{4}{Max-Planck-Institut f\"ur Astronomie, K\"onigstuhl 17, D-69117 Heidelberg, Germany}
\altaffiltext{5}{IoA}
\altaffiltext{6}{UW}
\email{david.hogg@nyu.edu}

\begin{abstract}
% Context:
The chemical-tagging hypothesis is that every star ought to carry---%
in its detailed chemical abundances---%
an informative signature of its molecular-cloud birth location and time.
This idea has not previously yielded scientific successes, probably because of the
noise in chemical-abundance measurements.
However, we have succeeded in improving vastly spectroscopic measurements with \thecannon\ 
(now delivering individual abundances with precisions around 0.04-dex).
% Aims:
We test the chemical-tagging hypothesis by looking at clusters in abundance space
and confirming that they are clustered in phase space.
% Methods:
We identify (by the k-means algorithm) overdensities of stars observed by the \apogee\ spectroscopic survey
in a 15-dimensional chemical-abundance space delivered by \thecannon,
and plot the associated stars in phase space.
We use \emph{only} abundance-space information (no positional information) to identify stellar groups.
% Results:
We find that clusters in abundance space are indeed clusters in phase space.
We recover some known clusters and find other interesting structures.
This confirms the chemical-tagging hypothesis and verifies the precision of abundances delivered by \thecannon.
This is the first-ever project to identify phase-space structures by blind search in abundance space;
the prospects for future data sets are very good.
\end{abstract}

\keywords{
  Galaxy: abundances
  ---
  Galaxy: stellar content
  ---
  Galaxy: structure
  ---
  globular clusters: general
  ---
  open clusters and associations: general
  ---
  stars: abundances
}

\section{Introduction}\label{sec:intro}

If every star in the Milky Way was born in a molecular cloud that has
unique chemical-abundance characteristics (as a function of time), and
if every star preserves its surface abundances (or significant aspects
of them) over its lifetime, then stars with common birthplaces and
time ought to be identifiable in their surface abundances.
That is, it ought to be possible to use stellar abundances to link up
co-eval stars---even if at the present day they are no longer close to
one another in phase space---and infer a detailed star-formation and
dynamical history for the Galaxy.
This idea has been dubbed ``chemical tagging''.
It was proposed long ago \citep{Freeman2002} and is one of the
principal motivations for a number of surveys, most notably
\apogee\ \citep{apogee}, \gaiaeso\ \citep{gaiaeso} and \galah\ \citep{galah13, galah15}.
In order to measure chemical tags and infer the Galaxy's dynamical
properties and formation history, these surveys are each measuring high resolution,
high signal-to-noise spectra for hundreds of thousands of stars across
the Galaxy's disk, bulge and halo.

Chemical tagging holds the promise of revealing not just the
star-formation history of the Galaxy, but also the accretion history
(as things that fall in are expected to be chemically distinct from
those that form in the parent body; \citealt{something}) and
stellar-orbit diffusion process like radial mixing and radial
migration (for example, \citealt{Roskar2009, Quillen2015}).  After
stars are born---or after a star cluster is accreted and
disrupted---associations or groups will disperse, through either
coherent orbital evolution or else incoherent diffusion, but the
chemical-abundance tags can in principle make the former members
identifiable as co-eval stellar siblings.

Although undeniably promising---and although it has motivated the
launch of expensive, large-scale spectroscopic surveys---the
chemical-tagging idea has yet to yield scientific fruit.
Part of the reason is that the level of chemical specificity required
is very high:
If there are thousands of (relevant) molecular clouds forming stars in
the recent history of the Milky Way, it will take many abundances,
each measured accurately, to make a sufficiently unique fingerprint
that stellar sibling identification will be reliable.

Stellar spectroscopic surveys now have the resolution, signal-to-noise,
wavelength coverage, and sample sizes to deliver chemical tags.
The issue at present is measurement precision (and accuracy).
The current precision on abundance measurements is not high enough
(for example, \citealt{Ting2015, Martel2015}).
That said, it is clear that the \emph{data} are precise enough;
there is enough signal-to-noise at the relevant locations in spectrum
space to deliver high-precision tags.
Our view is that the traditional methods for making abundance
measurements just aren't mature enough:
There are issues with the physical models, and the data-analysis
methods do not make use of all of the information in the data sets.
Improving the models and exploiting the entire information
content in the data is critical if we are going to deliver useful
chemical tags.

One note on accuracy and precision:

In principle, the 

These have not successfully identified components of the galaxy's
assembly that had distinct formation sites (\eg, Ting et al., 2015, Mitschang. et al. 2013,
Blanco-Cuaresma2015).

Blanco-Cuaresma et al., 2015 demonstrate that there is a large degree
of overlap when exploring the chemical space of known open clusters
which challenges chemically tagging and separating them, but also note
that there is room for improvement if more elements were included in
their analysis and models improved.
Ting et al., 2015 presented an algorithm to detect groups of
chemically homogeneous stars which they reported as challenging and
with a low signal to noise in the APOGEE dataset.
Mitschang et al., 2013 propose that chemical tagging may be able to
identify coeval groups of stars but these are not necessarily from a
distinct origin of formation.

The by-hand study of solar twins, for a handful of stars, has shown
however that the chemical information in some elements offers powerful
signatures for chemical tagging (Melendez2014, Jofre2015) and PCA
approach of Ting et al., 2014 shows there are many independent
dimensions of information contained in the multi abundance phase space
being delivered by the large spectroscopic surveys (e.g. Ting et al.,
2014 report a dimensionality of 7 for GALAH survey).
These approaches are important to establish which elements should be
targeted in surveys.

DWH:  We successfully achieve chemical tagging and we do so because we are
working with a higher precision that previously and exploit the full
information in the data....

DWH: With \thecannon\ we achieve both higher precision and exploit the full
information content of the data.....cite Casey et al., 2016 Ness et
al., 2016 in prep = do you want me to write things about this...

DWH: We do this with the APOGEE survey which covers a much wider
spatial extent but GALAH will be much more powerful as they will
report 29 elemental abundances which carry a larger quantity of
nucleosynthetic information...

DWH: Open clusters look one-metal (CITE); globulars look peculiar
(CITE); but no-one has ever gone the other way!...

DWH: Fast intro to \apogee.

DWH: Fast intro to \thecannon.

DWH: BASH the \apogee\ discontiguous field positioning.

DWH: Think of this as a hypothesis test.

\section{Sample and abundances}

MKN: Details about \apogee\ data.

AC: Details about \thecannon\ and implementation.

DWH: Figure showing the full data set; discuss structure.

\section{Method and results}

Although only hints of small-scale clustering in the abundance space
are visible in \figurename~DWH, exploration of the data indicates that
known clusters do appear in the abundance space as over-densities.
This encourages us to look for over-densities automatically in the
abundance space and see if anything found that way would be over-dense
in phase space.
The simplest method for clustering points in $D$-dimensional space is
the \emph{k-means} algorithm (DWH CITE).
This algorithm is fast and performs well in practice in problems of
this nature.

DWH: Scaling of the data
DWH: number of clusters
DWH: restarts for local minima
DWH: Citation out to numpy and sklearn.

DWH: Reminder that we don't need the absolute best clustering to
demonstrate the hypothesis.

DWH: Visualize densest clusters.  Some of them are known clusters.
Show examples.

DWH: Do we have streams or other structures?

DWH: What have we demonstrated?

\section{Discussion}

DWH: For the first time, we see small-scale informative features in abundance space.

DWH: When we identify the small-scale features in abundance space, they are informative in position space.

DWH: How do we show that we aren't just getting clusters because \apogee\ has plate-to-plate calibration variations?

DWH: Chemical tagging will work.

\acknowledgements
DWH: It is a pleasure to thank...

DWH: grants and grant numbers.

DWH: SDSS-III and SDSS-IV.

DWH: NASA ADS

DWH: github repo and hash.

DWH: Be sure to cite software a la the new ApJ guidelines.

\end{document}
