% This file is part of the Platypus project.
% Copyright 2016 the authors.

% # to-do list
% - put in figures showing the G vectors.

\documentclass[12pt, preprint]{aastex}

% text commands
\newcommand{\equationname}{equation}
\newcommand{\acronym}[1]{{\small{#1}}}
\newcommand{\project}[1]{\textsl{#1}}
\newcommand{\apogee}{\project{\acronym{APOGEE}}}

% math commands
\newcommand{\solar}{\odot}
\newcommand{\abundance}[1]{\lbrack{#1}\rbrack}
\newcommand{\element}[1]{\mathrm{#1}}
\newcommand{\given}{\,|\,}
\newcommand{\normal}{\mathcal{N}}


\begin{document}

\title{No straightforward nucleosynthesis scenario can explain the
  observed diversity of stellar abundances}
\author{DWH, JR, HWR, others}

\begin{abstract}
The elements heavier than Li are believed to be produced in stars and
supernovae, and delivered to the interstellar medium by supernova
explosions and stellar winds.
There are finite supernova and wind channels: Chemical enrichment is
expected to be dominated by core-collapse supernovae, type Ia
supernovae, and AGB stars.
If this is correct, and even if there are substantial mass- and
metallicity-dependent yields, it is impossible for even very
permissive nucleosynthesis models to produce a high dimensional family
of chemical abundance patterns in stars.
Here we compare the output of a huge diversity of nucleosynthetic
scenarios (varying star-formation histories, infall and outflow
parameters, and finite-numbers effects) to chemical abundances
observed in the \apogee\ data.
Although we are liberal with models, we restrict our consideration of
the data to red-clump stars, so we aren't comparing chemical abundance
measurements in very different kinds of stars.
We show that---from a theoretical perspective---there are directions
in chemical-abundance space that are forced to have vanishing
diversity, but that these same dimensions---in the data---\emph{do}
show significant diversity.
We also show that no reasonable adjustment to yield tables can resolve
this discrepancy; there is significant variance in the data along too
many axes.
There either must be other important processes (such as spallation), or
diversity in depletion of elements onto dust, or stochasticity in
supernova and wind yields, or chemical separation in ejecta and winds.
These results provide support and hope for the idea of \emph{chemical
tagging} as a tool to understand common stellar origins.
\end{abstract}

\keywords{
  ---
  World: Hello
  ---
}

\section{Introduction}

Nucleosynthesis is important.  Models are (at least slightly) wrong.
Data are now abundant. People have tried to measure dimensionality; that
isn't a well-posed question.

%\acknowledgements
%We thank Yuan-Sen Ting (Harvard) for valuable discussions.

\end{document}
